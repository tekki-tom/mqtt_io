\documentclass[a4paper,10pt]{scrartcl}
\usepackage[utf8]{inputenc}
\usepackage{listings}
\usepackage{graphicx}
\usepackage{hyperref}
\lstset{numbers=left, numberstyle=\tiny, stepnumber=2, numbersep=5pt, language = html}
% Title Page
\title{}
\author{}


\begin{document}
\maketitle

\begin{abstract}
\end{abstract}

\section{Systemaufbau}
Das System ist auf einer Experimentiersteckbrett (Breadboard) aufgebaut.

Der IO expander ist ein Microchip MCP 23017. Die Adresse ist auf 0 (0x20) festgelegt (A0 bis A2 auf GND). Die Bus-Leitungen (SCK, SDA) sind am MCP 23017 nicht mit pull\_ups auf VCC gezogen. Am Pico sind die internen pullup-Widerstände aktiviert (siehe Listing\ref{listing:PicoInit}.)

An b0 ist ein BC550 angeschlossen, daran ein Finder 52.09.012 Relais. Das Relais ist an +12V (a0) angeschlossen, a1 ist am Kollektor des BC550 angeschlossen, der Emitter des BC550 ist an GND.

Die Tests sollen einen Einsatz als IO-Expander für einen Raspberry Pi vorbereiten.

Interessante Infos sind hier: https://netzmafia.ee.hm.edu/skripten/hardware/RasPi/Projekt-I2C-Expander/index.html

\section{Test Pi Pico}
Als Testgerät wird ein Raspberry Pi Pico W eingesetzt.

Der Extender ist an Pin 38 mit GND verbunden, Pin 36 mit 3,3V.

Das Testprogramm initialisiert den Bus i2c an den default-Pins (CSK an Pin 7 / GP5, SDA an Pin 6 / GP4).

allgemeine Inits:
\begin{lstlisting}[float,caption=Includes (Pi Pico),captionpos=b,label=listing:PicoInclude]
#include <stdio.h>
#include "pico/stdlib.h"
#include "hardware/i2c.h"
#include "pico/cyw43_arch.h"
\end{lstlisting}

in main:
\begin{lstlisting}[float,caption=Variablen für Kommunikation,captionpos=b,label=listing:PicoVars]
uint8_t src[2];
uint8_t result = 2;
\end{lstlisting}

\begin{lstlisting}[float,caption=init I2C-Bus,captionpos=b,label=listing:PicoInit]
i2c_init(i2c_default, 100 * 1000);// 48000);
gpio_set_function(PICO_DEFAULT_I2C_SDA_PIN, GPIO_FUNC_I2C);
gpio_set_function(PICO_DEFAULT_I2C_SCL_PIN, GPIO_FUNC_I2C);
gpio_pull_up(PICO_DEFAULT_I2C_SDA_PIN);
gpio_pull_up(PICO_DEFAULT_I2C_SCL_PIN);
const int io_expander0 = 0x20;
\end{lstlisting}

% \captionof{lstlisting}{initialisierung des I2C-Bus am Pi Pico}

Anschließend sendet das Programm die Befehlsfolge:
\begin{lstlisting}[float,caption=Init der Ports a und b auf Ausgang,captionpos=b]
src[0] = 0x00;
src[1] = 0x00;
result = i2c_write_blocking(i2c_default, io_expander0, src, 2, false);
src[0] = 0x01;
src[1] = 0x00;
result = i2c_write_blocking(i2c_default, io_expander0, src, 2, false);
\end{lstlisting}

und initialisiert damit die Ports a und b als Ausgänge (alle Pins).
mit \lstinline|printf("i2c result: %d\n", result);| wird via Terminal ( minicom -b 115200 -o -D /dev/ttyACM0 ) die Rückmeldung des i2c-Bus ausgegeben. 2 = okay, 254 = nok.

In einer Endlosschleife werden dann die Pins a0 und b0 wechselnd auf 1 und 0 gesetzt:
\begin{lstlisting}[float,caption=Endlosschleife. Schaltet a0 und b0 zyklisch ein und aus,captionpos=b]while (true)
  {
      src[0] = 0x15;
      src[1] = 0x00;
      result = i2c_write_blocking(i2c_default, io_expander0, src, 2, false);
      src[0] = 0x14;
      src[1] = 0x01;
      result = i2c_write_blocking(i2c_default, io_expander0, src, 2, false);

      printf("i2c led on  result: %d\n", result);

      gpio_put(LED_PIN, 1); // IO an pico
      cyw43_arch_gpio_put(CYW43_WL_GPIO_LED_PIN, 1); // LED des pico, Modell w
      sleep_ms(1000);
      src[0] = 0x15;
      src[1] = 0x01;
      result = i2c_write_blocking(i2c_default, io_expander0, src, 2, false);
      src[0] = 0x14;
      src[1] = 0x00;
      result = i2c_write_blocking(i2c_default, io_expander0, src, 2, false);

      printf("i2c led off result: %d\n", result);
      gpio_put(LED_PIN, 0);
      cyw43_arch_gpio_put(CYW43_WL_GPIO_LED_PIN, 0);
      sleep_ms(1000);
      printf("blink\n");
  }
\end{lstlisting}
Das Ergebnis auf dem Bus zeigt Bild \ref{abb:PiPicoBlink01}. Im Bild sind die Register 0x12 und 0x13 adressiert, dies ist im Script in 0x14 und 0x15 korrigiert.
\begin{figure}[htb]
 \includegraphics[width=\textwidth]{pi\_a0\_b0\_blink.png}
 % pi_a0_b0_blink.png: 0x0 px, 300dpi, 0.00x0.00 cm, bb=
 \caption{Ausgabe des I2C-Bus (Adr. 0x20, Register 0x13 und 0x12 sind im Code auf 0x15 und 0x14 korrigiert)}
 \label{abb:PiPicoBlink01}
\end{figure}

Der Test war überwiegend erfolgreich verlaufen. Wenn durch einen Wackelkontakt Pin SCK oder SDA unterbrochen wurde, wurde der Kontakt nicht wieder hergestellt. Es musste durch trennen der Versorgung des Pico und ca. 30 - 60 sec. Warten das System neu initialisiert werden.
\subsection{Zweiter Test Pi Pico}
Testaufbau: Sender Pi Pico, Programm wie oben. Empfänger: 2x MCP 23017. Ein Expander (links) auf Adr. 0x20, der andere auf Adr. Oszilloskop
\begin{description}
 \item[Kanal 1] SCK an Pico,
 \item[Kanal 2] SDA an Pico,
 \item[Kanal 3] SCK an Expander links (0x22),
 \item[Kanal 4] SDA an Expander links (0x22)
\end{description}
Der Expander links lässt sich nicht korrekt ansprechen. Das Signal am Pico ist in Ordnung, beim Signal am Expander links fehlt bei einigen Tests SCK.

Einspeisung erfolgt zuerst am Expander links. Dann am Expander rechts. Bei Einspeisung rechts ist links ist zeitweise kein SCK messbar.

Nach Vertauschen der Reihenfolge im Programm (zuerst an 0x22, dann 0x20) reagiert der Expander 0x22 auch nicht. siehe Bild \ref{abb:0x20ok0x22nok}

\begin{figure}[htb]
 \includegraphics[width=\textwidth]{pico\_2exp\_send\_0x22\_nok\_0x20\_ok\_mark.png}
 % pico_2exp_send_0x22_nok_0x20_ok_mark.png: 800x480 px, 72dpi, 28.22x16.93 cm, bb=0 0 800 480
  \caption{Zwei Expander. Sende Reihenfolge 0x22, dann 0x20}
 \label{abb:0x20ok0x22nok}
\end{figure}


Bei beiden Expandern /Reset auf VCC gezogen, das behebt das Problem, von undefinierten ausfällen. Vorher war /Reset nicht beschaltet.

\section{Test Pi 3b+}
% Beim Raspberry Pi (Modell 3b+, OS: Raspian) wurde der i2c-Bus aktiviert.

Als Pi wird ein Raspberry Pi 3b+ mit Raspian verwendet. Am Pi ist der i2c Bus aktiviert.
zum Testen ist der Takt des i2c auf 1kHz begrenzt. Bei Messung mit dem Oszilloskop (Rigol DS 1054Z) ergibt sich, dass der Takt bei 100kHz bleibt (siehe Bild \ref{abb:PiDetectGood}).

Ein Test mit \lstinline|i2cdetect -y 1 0x20 0x27| ergab folgende Situation: siehe Bild \ref{abb:PiDetectGood}
\begin{figure}[htb]
 \includegraphics[width=\textwidth]{rpi_no_pull_up_detect_0x20_0x27.png}
 % rpi_no_pull_up_detect_0x20_0x27.png: 0x0 px, 300dpi, 0.00x0.00 cm, bb=
 \caption{detect vom Raspberry Pi, Bereich 0x20 bis 0x27. Der Expander ist hier auf 0x20 eingestellt.}
 \label{abb:PiDetectGood}
\end{figure}
 An der Adresse 0x20 antwortete der IC planmäßig. An Adresse 0x21 kam keine Antwort (gekennzeichnet durch das "`?"') danach bricht das Signal unerwartet zusammen. Diese Zusammenbrüche sind bei Test mit dem Raspberry Pi mehrfach festzustellen, eine Ursache ist mir noch nicht bekannt. Ein erneuter Test mit detect ergab Bild \ref{abb:PiDetect2Fail}. Der Expander konnte die Adresse nicht dekodieren und der Detect schlug damit fehl.

 \begin{figure}[htb]
 \includegraphics[width=\textwidth]{rpi\_no\_pull\_up\_detect\_2\_fail.png}
 % rpi_no_pull_up_detect_2_fail.png: 0x0 px, 300dpi, 0.00x0.00 cm, bb=
 \caption{Detect am Pi. Signal ist nicht korrekt auf 0V gezogen.}
 \label{abb:PiDetect2Fail}
\end{figure}

\end{document}
